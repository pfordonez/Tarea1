\documentclass[12pt,a4paper]{article}
\usepackage[utf8]{inputenc}
\usepackage[spanish]{babel}
\usepackage{amsmath}
\usepackage{amsfonts}
\usepackage{amssymb}
\usepackage{hyperref}
\usepackage{graphicx}

% cabecera y pie
\usepackage{fancyhdr} % activamos el paquete
\pagestyle{fancy}
\usepackage[margin=2cm, headheight=10pt, includeheadfoot]{geometry}
\chead{\includegraphics[width=\textwidth]{logo}}
\renewcommand{\headrulewidth}{0pt}
%\renewcommand{\headrulewidth}{0pt}
%\renewcommand{\footrulewidth}{0pt}
%\setlength\headheight{80.0pt}
%\addtolength{\textheight}{-80.0pt}
%\chead{\includegraphics[width=\textwidth]{logo}}
%\renewcommand{\headrulewidth}{0pt}
\title{Ejercicio1b}
\author{Pablo Fernando Ordo<U+00F1>ez Ordo<U+00F1>ez}

\usepackage{Sweave}
\begin{document}
\Sconcordance{concordance:Pablo_Ordonez_Ej1b.tex:Pablo_Ordonez_Ej1b.Rnw:%
1 24 1 1 0 2 1 1 5 56 1}



\maketitle
\thispagestyle{fancy}

\section*{Velocidad de ca<U+00ED>da libre de la pelota de ping-pong}
En este ejercicio consideraremos el ejemplo de una pelota de ping pong que se deja caer bajo la influencia de la fuerza de la gravedad y la resistencia del aire.
La evoluci<U+00F3>n de la velocidad de la pelota a lo largo del tiempo se puede modelar por medio de la siguiente ecuaci<U+00F3>n diferencial, donde la velocidad $v$ es la variable de estado, y $g = 9,8m/seg^{2}$  es la aceleraci<U+00F3>n de la gravedad, $M = 0,0028kg$ es la masa de la pelota, $R = 0,02m$ es su radio, $\mu = 1,789 \times 10^{-5} kg/(m seg)$ es la viscosidad din<U+00E1>nica del aire, y $\rho = 1,205 kg/m^{3}$ su densidad:

\begin{equation}
\frac{dv}{dt}=-g-\frac{\rho\pi R^{2}C_{D}(Re)}{2M}\mid v \mid v,    v(0)=0,
\end{equation}
donde:
\begin{align*}
Re = \frac{2\rho R}{\mu} \mid v \mid,  
C_{D}(Re) =
\left\lbrace
\begin{array}{ccc}
\frac{24}{Re}+\frac{2}{5}+\frac{6}{1+\sqrt{Re}} &\textup{si } &Re > 0, \\
0 &\textup{si } &Re = 0. 
\end{array}
\right.
\end{align*}

Aqu<U+00ED>, $C_{D}(Re)$ es el coeficiente de arrastre de un fluido cualquiera sobre una esfera de superficie suave, que es funci<U+00F3>n del $n\acute{u}mero\:de\:Reynols\:Re$, directamente proporcional a la magnitud de la velocidad $|v|$ de la esfera. Las unidades utilizadas son metros para la longitud, segundos para el tiempo, y kilogramos para la masa.
\begin{enumerate}
\item 
Aplicar el m<U+00E9>todo Euler con longitud de paso $h = 1/200$ para obtener aproximaciones $v_j$ de la velocidad $v(t_j)$ de la pelota (donde $v(t)$ es la soluci<U+00F3>n del problema (1) en los tiempos
\begin{equation}
t0 = 0, t1 = h, t2 = 2h,...,t3998 = 3998h, t3999 = 3999h, t4000 = 4000h = 20. 
\end{equation}

\item
Comprobar que la velocidad en los tiempos finales (por ejemplo, en los <U+00FA>ltimos seis tiempos) es ex<U+00E1>tamente la misma. Esta velocidad es la que se denomina velocidad terminal, que denotaremos por $v_T$. Para dicha velocidad $v_T$, la aceleraci<U+00F3>n de la gravedad y la resistencia del aire tienen la misma magnitud y signos opuestos, es decir, el lado derecho de la ecuaci<U+00F3>n diferencial (la aceleraci<U+00F3>n total) se anula cuando $v$ toma el valor de la velocidad terminal. Comprobar si esto es efectivamente as<U+00ED>, y comentar el resultado.

\item
El modelo matem<U+00E1>tico (1) admite la siguiente versi<U+00F3>n simplificada (estudiada en la presentaci<U+00F3>n de la primera parte del Tema 1, y considerada tambi<U+00E9>n en el Ejercicio 1.a)
\begin{equation}
\frac{dv}{dt} = -g+\frac{c}{M}v^{2},\; v(0) = 0,
\end{equation}
donde $c > 0$ es un par<U+00E1>metro constante a determinar. Un criterio razonable para determinar el valor de $c$ es exigir que la velocidad terminal en el modelo simplificado coincida con la velocidad $v_T$ obtenida en el apartado anterior. Determinar el valor de $c$ que verifica dicho criterio.

\item
El modelo simplicado (3) tiene la ventaja de que conocemos explitamente la solucion exacta. Para hacernos una idea del error que hemos cometido al aproximar la soluci<U+00F3>n del problema (1) por medio del m<U+00E9>todo de Euler, aplicaremos dicho m<U+00E9>todo al problema simplificado (3) para la misma discretizacion temporal (2), para obtener las aproximaciones $\tilde{v}_j \approx v(t_j)$, donde $v(t)$ es la soluci<U+00F3>n exacta del problema (3). Incluiremos en una misma figura, (i) el error $|\tilde{v}_j - v(t_j)|$ cometido en funci<U+00F3>n de los tiempos $t_j$ al aproximar la soluci<U+00F3>n de (3) con el m<U+00E9>todo de Euler para la discretizaci<U+00F3>n (2), y (ii) la diferencia $\mid\tilde{v}_j - v_j\mid$ entre los resultados num<U+00E9>ricos (obtenidos con el m<U+00E9>todo de Euler) para los problemas (1) y (3) respectivamente.

\item
Si se busca en internet informaci<U+00F3>n en sobre la velocidad terminal de una pelota de ping-pong, se puede ver que en algunos casos se habla de $9,5m/seg$. V<U+00E9>ase por ejemplo el resumen de un articulo cient<U+00ED>fico de 1984 que se puede encontrar en el sitio http://aapt.scitation.org/doi/10.1119/1.13904. Comprobar hasta que punto coincide esto con el resultado obtenido en el apartado 2. En la actualidad, el radio de una pelota de ping-pong estandar es $R = 0,02 m$ y su masa $M = 0,0028 kg$, sin embargo, antes del ano 2000, las pelotas de ping-pong sol<U+00ED>an ser algo menores. En el articulo mencionado, consideran pelotas de radio $R = 0,0185m$ y masa $M = 0,00266kg$. Repetir la simulaci<U+00F3>n del apartado 1. pero con $R = 0,0185m$ y $M = 0,00266kg$. Representar en una misma figura los resultados de la simulaci<U+00F3>n (con gr<U+00E1>ficas de la velocidad con respecto al tiempo) obtenidos en el apartado 1 (para $R = 0,02m$ y $M = 0,0028 kg$) y en este apartado (para los datos de la pelota de ping-pong del siglo pasado). <U+00BF>Son las diferencias apreciables? <U+00BF>En cual de los dos casos es la velocidad terminal mas cercana a $9,5 m/seg$?

\item
Los valores $\mu = 1,789 \times 10^{-5} kg/(m seg)$ de la viscosidad din<U+00E1>nica del aire y de su densidad $\rho = 1,205 kg/m^{3}$ dados al principio del enunciado de este problema se refieren a los valores estandar de la atm<U+00F3>sfera al nivel del mar. En cambio, a $3000 m$ de altura, $\mu = 1,69 \times 10^{-5} kg/(m seg),\, \rho = 0,9 kg/m^{3}$. Repetir la simulaci<U+00F3>n del apartado 1. (con $R = 0,02 m$) pero con estos valores de los par<U+00E1>metros. Representar en una misma figura los resultados de la simulaci<U+00F3>n de las velocidades al nivel del mar y a $3000 m$ de altura. <U+00BF>Son las diferencias apreciables?

\end{enumerate}



\end{document}


